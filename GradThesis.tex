% !Mode:: "TeX:UTF-8"
%!TEX program = xelatex
\expandafter\def\csname CTEX@spaceChar\endcsname{\hspace{1em}} %add by x
\documentclass[twoside,openany]{SDUthesis}

% 控制位于页面顶部或底部的表格与正文的距离为 5pt
\setlength{\textfloatsep}{5pt plus 1pt minus 2pt}

% 控制位于页面内部的表格与正文的距离为 5pt
\setlength{\intextsep}{5pt plus 1pt minus 2pt}

%改动:% 在此添加 geometry 设置 改成A4要求
\usepackage[a4paper, top=2.8cm, bottom=2.5cm, left=2.5cm, right=2.5cm]{geometry}

%  原本的
% \newcommand{\makeheadrule}{%页眉设置
%  \makebox[0pt][l]{\hspace{-0.2cm}\rule[0.55\baselineskip]{\linewidth}{0.6pt}}% 下线设置
%  \rule[0.7\baselineskip]{\headwidth}{0pt}} %上线设置

% 重定义 \makeheadrule 以实现奇数页平移,偶数页不平移
\newcommand{\makeheadrule}{%
    \ifodd\value{page}  % 如果是奇数页
        \makebox[0pt][l]{\rule[0.55\baselineskip]{\linewidth}{0.6pt}} % 向左平移
    \else  % 如果是偶数页
        \makebox[0pt][l]{\hspace{-0.2cm}\rule[0.55\baselineskip]{\linewidth}{0.6pt}}  % 不平移
    \fi
    \rule[0.7\baselineskip]{\headwidth}{0pt}  % 上线
}
 
\renewcommand{\headrule}{%
{
 \if@fancyplain\let\headrulewidth\plainheadrulewidth\fi
  \makeheadrule}}
\makeatother


% \usepackage{amsmath}
\renewcommand\theequation{\arabic{chapter}-\arabic{equation}}
\usepackage{booktabs}
\usepackage{pifont}
\usepackage{gbt7714}
\renewcommand\thefootnote{\ding{\numexpr171+\value{footnote}}}
\usepackage{threeparttable}



\makeatletter
\newcommand{\figcaption}{\def\@captype{figure}\caption}
\newcommand{\tabcaption}{\def\@captype{table}\caption}
\makeatother

\setlength{\abovecaptionskip}{0pt}
\setlength{\belowcaptionskip}{0pt}

\usepackage[figuresright]{rotating} %add by x
\usepackage{arydshln} %add by x
\usepackage[OT1]{fontenc} %add by x
\setmainfont{Times New Roman} %add by dss

\newcommand{\PreserveBackslash}[1]{\let\temp=\\#1\let\\=\temp} %add by x
\newcolumntype{C}[1]{>{\PreserveBackslash\centering}p{#1}}
\newcolumntype{R}[1]{>{\PreserveBackslash\raggedleft}p{#1}}
\newcolumntype{L}[1]{>{\PreserveBackslash\raggedright}p{#1}}


\usepackage{mathrsfs}
\usepackage{pdfpages}
\usepackage{changepage}%160311整段缩进

\usepackage{color}
\usepackage{ulem}
\usepackage{graphicx}
% \usepackage{subfig}
\usepackage{longtable}
\usepackage{epstopdf}

\usepackage{bicaption}
\usepackage[justification=centering]{caption} % 实现中英文图表的标题居中
\usepackage{subcaption}

\usepackage{bm}
\usepackage{bbm}

\usepackage{multirow}
\usepackage{bbding}


\captionsetup[figure][bi-second]{name=Fig.}
\captionsetup[table][bi-second]{name=Tab.}

\newtheorem{lemma}{引理}
\newtheorem{theorem}{定理}
\setlength{\bibsep}{0.3pt} 

\begin{document}
\raggedbottom

% 封面
% \includepdfmerge{./Chapters/封面.pdf,-} 
\includepdfmerge{./Chapters/封面-非匿名.pdf,-}
% \includepdfmerge{./Chapters/封面扉页.pdf,-}
\includepdfmerge{./Chapters/原创性声明扫描件.pdf,-} 
%需要等学院下发titlePage,然后添加 by susie

% 版权声明
% \makeOSandCPRTpage %提交时要加
% \includepdfmerge{copyrights.pdf,-}

%%%%%%%%%%%%%%%%%%%%%%%%%%%%%%
%% 论文部分开始
%%%%%%%%%%%%%%%%%%%%%%%%%%%%%%
\SDUfrontmatter

%%%%%%%%%%%%%%%%%%%%%%%%%%%%%%
%% 中文摘要
%%%%%%%%%%%%%%%%%%%%%%%%%%%%%%
\begin{abstract}
\eabstract{Chinese Abstract}

近来年,XXX领域受到广泛关注。本论文针对XXX领域中现有方法存在的YYY问题进行分析,并提出两种改进方法来解决这些问题。。。。。。。


\vspace{0.5cm}
\noindent\textbf{关键词:} 奥赛;刚就;入团;催催;憨憨
\end{abstract}

%%%%%%%%%%%%%%%%%%%%%%%%%%%%%%
%% 英文摘要
%%%%%%%%%%%%%%%%%%%%%%%%%%%%%%
\begin{englishabstract}
\eabstract{English Abstract}

\vspace{0.5cm}
\noindent\textbf{Keywords:} Iksadjk; Afgkfg; Hkjxc; Hhic;

\end{englishabstract}

%%%%%%%%%%%%%%%%%%%%%%%%%%%%%%
%% 目录页
%%%%%%%%%%%%%%%%%%%%%%%%%%%%%%
\SDUcontents
\SDUEcontents

%%%%%%%%%%%%%%%%%%%%%%%%%%%%%%
%% 正文内容部分开始
%%%%%%%%%%%%%%%%%%%%%%%%%%%%%%
\SDUmainmatter
\chapter{绪论}
\echapter{Introduction}
\label{chap1}
\section{研究背景及意义}
\esection{Research Background and Motivations}
XXXX YYYY ZZZZ  引用现有文献示例:XXX等人\cite{mittal2022survey}发现,洗澡要用热水。
\chapter{相关研究理论概述}
\echapter{Background Knowledge}
\label{chap2}

\section{卷积神经网络}
\esection{Convolutional Neural Networks}
卷积神经网络 。。。。。。
\chapter{基于类激活图增强的修正梯度方法}
\echapter{CAM-Enhanced Rectified Gradient}
\label{chap3}

\section{引言}
\esection{Motivation}
引言 。。。。



\section{问题分析}
现有问题存在的问题


\section{提出方法}
提出新的方法解决现有问题
\chapter{基于类激活图增强的积分梯度方法}
\echapter{CAM-Enhanced Integrated Gradients}
\label{chap4}

\newcommand{\diff}{\mathop{}\!\mathrm{d}} % 自定义命令简化积分中d的使用

\section{引言}
\esection{Motivation}

本章针对XX问题做出了YY改进,提出了新的ZZ方法来实现这一点。
\chapter{总结与展望}
\echapter{Conclusions and Future Works}
\label{chap5}


\section{本文总结}
\esection{Conclusions}
本论文针对XX问题进行了研究,分析出现有方法存在的问题并提出两种方法来解决问题,本文的具体改进如下:

1. XXX

2. YYY
% \chapter{附录}
\echapter{Appendix}
\label{chap6}

本章对工作\ref{chap4}中的学习框架Meta-T提供了理论性分析。

\section{光滑性证明}
\esection{Proof of Smoothness}
\label{chap6_1}

给定一个大小为$n$的元数据集$\{(\mathbf{x}_1^l, \mathbf{y}_1^l),...,(\mathbf{x}_n^l, \mathbf{y}_n^l)\}$,本文学习框架中的元损失可以写成:
\begin{equation}
    \label{eq:app_1}
         L_{\rm meta}(\mathbf{w}^*({\Theta}))  =
     \frac{1}{n} \sum\limits_{i=1}^n H(\mathbf{y}_i^l, f(\mathbf{x}_i^l; \mathbf{w}^*({\Theta}))),
\end{equation}
给定一个无标签的数据集$\{\mathbf{x}_1,...,\mathbf{x}_{(\mu \times n)}\}$,其大小为$\mu \times n$,本文学习框架中的训练损失可以写成:
\begin{equation}
    \begin{aligned}
        \label{eq:app_2}
        L_{train}(\mathbf{w},\Theta) &= L_u(\mathbf{w},\Theta) = 
        \frac{1}{n\mu} \sum\limits_{i=1}^{n\mu} \ell_{\mathbf{x}_i}(\mathbf{w}, \Theta) \\ &=
        \frac{1}{n\mu} \sum\limits_{i=1}^{n\mu} \mathbbm{1}(\max(f(\mathcal{A}^w(\mathbf{x}_i; \mathbf{w}))) > \mathcal{V}_i(\mathbf{w}, \Theta))  \cdot   H(\hat{\mathbf{y}}_i, f(\mathcal{A}^s(\mathbf{x}_i; \mathbf{w}))),
    \end{aligned}
\end{equation}
其中$\mathbf{\hat y}_i$是模型对样本$\mathbf{x}_i$弱增强版本的预测结果,且$\mathcal{V}_i(\mathbf{w}, \Theta) = \mathcal{V}(g(f(\mathbf{x}_i; \mathbf{w})), \overline {\rm p}_c^t; \Theta)$。在实际训练过程中,采用了替代函数$\mathcal{S}(x) = \frac{1}{1 + \exp^{- \beta x}}$来替代指示函数,因此公式 ~(\ref{eq:app_2})可以写成:
\begin{equation}
    \label{eq:app_2new}
    L_{train}(\mathbf{w},\Theta) = \frac{1}{n\mu} \sum\limits_{i=1}^{n\mu} \ell_{\mathrm{x}_i}(\mathbf{w}, \Theta) =
    \frac{1}{n\mu} \sum\limits_{i=1}^{n\mu} \mathcal{S}_i(\mathbf{w}, \Theta) \cdot   H(\hat{\mathbf{y}}_i, f(\mathcal{A}^s(\mathbf{x}_i; \mathbf{w}))).
\end{equation}
其中,替代函数$\mathcal{S}(\cdot)$的输入为$(\max(f(\mathcal{A}^w(\mathbf{x}_i; \mathbf{w}))) > \mathcal{V}_i(\mathbf{w}, \Theta))$。

\begin{proof}
首先,公式~(\ref{eq:step2})中TGN网络参数$\Theta$更新的反向传播可以写成:
\begin{equation}
\begin{aligned}
\label{eq:app_eq_smooth}
     &   \frac{1}{n} \sum\limits_{i=1}^{n} \nabla_{\Theta} H_i^{\rm meta}(\mathbf{\hat w}^{(t)}(\Theta)) \Big{|}_{\Theta^{(t)}} \\
     = & \frac{1}{n} \sum\limits_{i=1}^{n} \frac{\partial H_i^{\rm meta}(\mathbf{\hat w})}{\partial {\mathbf{\hat  w}}} \Big{|}_{\mathbf{\hat  w}^{(t)}} \sum\limits_{j=1}^{n\mu}
     \frac{\partial \mathbf{\hat w}^{(t)}(\Theta)}
            {\partial \mathcal{S}_j(\mathbf{w}^{(t)}; \Theta)} \,
     \frac{\partial \mathcal{S}_j(\mathbf{w}^{(t)}; \Theta)}
            {\partial \mathcal{V}_j(\mathbf{w}^{(t)}; \Theta)} \,
     \frac{\partial \mathcal{V}_j(\mathbf{w}^{(t)}; \Theta)}{\partial \Theta} \Big{|}_{\Theta^{(t)}} \\
     = & \frac{-\alpha}{n^2\mu} \sum\limits_{i=1}^{n} \frac{\partial H_i^{\rm meta}(\mathbf{\hat w})}{\partial {\mathbf{\hat  w}}} \Big{|}_{\mathbf{\hat  w}^{(t)}}  \sum\limits_{j=1}^{n\mu} \frac{\partial \ell_{\mathbf{x}_j}(\mathcal{S}_j(\mathbf{w}))}{\partial \mathcal{S}_j(\mathbf{w})} \, \frac{\partial \mathcal{S}_j(\mathbf{w})}{\partial \mathbf{w}}
     \Big{|}_{\mathbf{w}^{(t)}}
     \frac{\partial \mathcal{V}_j(\mathbf{w}^{(t)}; \Theta)}{\partial \Theta} \Big{|}_{\Theta^{(t)}} \\
     = & \frac{- \alpha}{n\mu} \sum\limits_{j=1}^{n\mu} 
     \bigg{(}
     \frac{1}{n} \sum\limits_{i=1}^{n} \frac{\partial H_i^{\rm meta}(\mathbf{\hat w})}{\partial {\mathbf{\hat  w}}} \Big{|}_{\mathbf{\hat  w}^{(t)}}^T \frac{\partial \ell_{\mathbf{x}_j}(\mathcal{S}_j(\mathbf{w}))}{\partial \mathcal{S}_j(\mathbf{w})} \, \frac{\partial \mathcal{S}_j(\mathbf{w})}{\partial \mathbf{w}} \Big{|}_{\mathbf{w}^{(t)}}
     \bigg{)} \frac{\partial \mathcal{V}_j(\mathbf{w}^{(t)}; \Theta)}{\partial \Theta} \Big{|}_{\Theta^{(t)}}.
\end{aligned}
\end{equation}
让$G_{ij} = \frac{\partial H_i^{\rm meta}(\mathbf{\hat w})}{\partial {\mathbf{\hat  w}}} \big{|}_{\mathbf{\hat  w}^{(t)}}^T \frac{\partial \ell_{\mathbf{x}_j}(\mathcal{S}_j(\mathbf{w}))}{\partial \mathcal{S}_j(\mathbf{w})} \, \frac{\partial \mathcal{S}_j(\mathbf{w})}{\partial \mathbf{w}} \big{|}_{\mathbf{w}^{(t)}}$,同时将$G_{ij}$带入到公式~(\ref{eq:app_eq_smooth})中,可得到
\begin{equation}
    \label{eq:eq_Gij}
    \Theta^{(t+1)} = \Theta^{(t)} + \frac{\alpha\beta}{n\mu}
    \sum\limits_{j=1}^{n\mu}
    \bigg{(}
    \frac{1}{n} \sum\limits_{i=1}^{n} G_{ij}
    \bigg{)}
     \frac{\partial \mathcal{V}_j(\mathbf{w}^{(t)}; \Theta)}{\partial \Theta} \Big{|}_{\Theta^{(t)}}.
\end{equation}
参数$\Theta$关于元损失的梯度可以写成:
\begin{equation}
        \begin{aligned}
        \label{eq:eq_firstOrder}
            & \nabla_{\Theta} H^{\rm meta}(\mathbf{\hat w}^{(t)}(\Theta)) \Big{|}_{\Theta^{(t)}}  \\
            = & -\frac{\alpha}{n\mu} \sum\limits_{j=1}^{n\mu} 
            \bigg{(}
                \frac{\partial H^{\rm meta}(\mathbf{\hat w})}{\partial {\mathbf{\hat  w}}} \Big{|}_{\mathbf{\hat  w}^{(t)}}^T \frac{\partial \ell_{\mathbf{x}_j}(\mathcal{S}_j(\mathbf{w}))}{\partial \mathcal{S}_j(\mathbf{w})} \, \frac{\partial \mathcal{S}_j(\mathbf{w})}{\partial \mathbf{w}} \Big{|}_{\mathbf{w}^{(t)}}
            \bigg{)}
            \frac{\partial \mathcal{V}_j(\mathbf{w}^{(t)}; \Theta)}{\partial \Theta} \Big{|}_{\Theta^{(t)}}.
    \end{aligned}
\end{equation}
令$\mathcal{V}_j(\Theta) = \mathcal{V}_j(\mathbf{w}^{(t)}; \Theta)$,同时引入共识~(\ref{eq:eq_Gij})中的$G_{ij}$。对等式~(\ref{eq:eq_firstOrder})两边同时求导,则有:
\begin{equation}
    \begin{aligned}
        \label{eq:eq_secondOrder}
        & \nabla_{\Theta^2}^2 H^{\rm meta}(\mathbf{\hat w}^{(t)}(\Theta)) \Big{|}_{\Theta^{(t)}}  \\
        = & -\frac{\alpha}{n\mu} \sum\limits_{j=1}^{n\mu}
        \bigg{[}
        \frac{\partial}{\partial \Theta} (G_{ij}) \Big{|}_{\Theta^{(t)}} \frac{\partial \mathcal{V}_j(\Theta)}{\partial \Theta} \Big{|}_{\Theta^{(t)}} 
        +
        (G_{ij}) \frac{\partial^2 \mathcal{V}_j(\Theta)}{\partial ^2 \Theta} \Big{|}_{\Theta^{(t)}}
        \bigg{]}.
    \end{aligned}
\end{equation}
公式~(\ref{eq:eq_secondOrder})右侧的第一项可以写成如下形式:
\begin{equation}
    \begin{aligned}
    \label{eq:app_second_firstTerm}
    & \left\|  
        \frac{\partial}{\partial \Theta} (G_{ij}) \Big{|}_{\Theta^{(t)}} \frac{\partial \mathcal{V}_j(\Theta)}{\partial \Theta} \Big{|}_{\Theta^{(t)}}   \right\|   \\
    \leq & \, \delta
    \left\|  
        \frac{\partial}{\partial \mathbf{\hat w}} 
        \bigg{(} 
            \frac{\partial H^{\rm meta} (\mathbf{\hat w})}{\partial \Theta} \Big{|}_{\Theta^{(t)}}
        \bigg{)} \Big{|}_{\mathbf{\hat w}^{(t)}}^T 
        \frac{\partial \ell_{\mathbf{x}_j}(\mathcal{S}_j(\mathbf{w}))}{\partial \mathcal{S}_j(\mathbf{w})} \, \frac{\partial \mathcal{S}_j(\mathbf{w})}{\partial \mathbf{w}} \Big{|}_{\mathbf{w}^{(t)}}     \right\| \\
    = & \, \delta
    \left\|
        \frac{\partial}{\partial \mathbf{\hat w}}
        \bigg{(} 
            \frac{\partial H^{\rm meta} (\mathbf{\hat w})}{\partial \mathbf{\hat w}} \Big{|}_{\mathbf{\hat w}^{(t)}} \,
            \frac{-\alpha}{n\mu} \sum\limits_{k=1}^{n\mu} 
            \frac{\partial \ell_{\mathbf{x}_k}(\mathcal{S}_j(\mathbf{w}))}{\partial \mathcal{S}_j(\mathbf{w})}
            \frac{\partial \mathcal{S}_j(\mathbf{w})}{\partial \mathbf{w}}
             \Big{|}_{\mathbf{w}^{(t)}} \frac{\partial \mathcal{V}_k(\Theta)}{\partial \Theta} \Big{|}_{\Theta^{(t)}}
        \bigg{)} \Big{|}_{\mathbf{\hat w}^{(t)}}^T 
         \frac{\partial \ell_{\mathbf{x}_j}(\mathcal{S}_j(\mathbf{w}))}{\partial \mathcal{S}_j(\mathbf{w})} \, \frac{\partial \mathcal{S}_j(\mathbf{w})}{\partial \mathbf{w}} \Big{|}_{\mathbf{w}^{(t)}}   \right\|   \\
    = & \, \delta
    \left\|
        \bigg{(} 
            \frac{\partial^2 H^{\rm meta} (\mathbf{\hat w})}{\partial \mathbf{\hat w}^2} \Big{|}_{\mathbf{\hat w}^{(t)}} \,
            \frac{-\alpha}{n\mu} \sum\limits_{k=1}^{n\mu} 
            \frac{\partial \ell_{\mathbf{x}_k}(\mathcal{S}_j(\mathbf{w}))}{\partial \mathcal{S}_j(\mathbf{w})}
            \frac{\partial \mathcal{S}_j(\mathbf{w})}{\partial \mathbf{w}}
             \Big{|}_{\mathbf{w}^{(t)}} \frac{\partial \mathcal{V}_k(\Theta)}{\partial \Theta} \Big{|}_{\Theta^{(t)}}
        \bigg{)} \Big{|}_{\mathbf{\hat w}^{(t)}}^T 
         \frac{\partial \ell_{\mathbf{x}_j}(\mathcal{S}_j(\mathbf{w}))}{\partial \mathcal{S}_j(\mathbf{w})} \, \frac{\partial \mathcal{S}_j(\mathbf{w})}{\partial \mathbf{w}} \Big{|}_{\mathbf{w}^{(t)}}  
     \right\|   \\
     \leq & \, 
     \alpha L \delta^2 \phi^2 \zeta^2,
    \end{aligned}
\end{equation}
因为$\left\| \frac{\partial H(\mathbf{\hat w})}{\partial {\mathbf{\hat  w}}} \big{|}_{\mathbf{\hat  w}^{(t)}}^T \right\| \leq \rho,
\left\| \frac{\partial \ell_{\mathbf{x}_j}(\mathcal{S}_j(\mathbf{w}))}{\partial \mathcal{S}_j(\mathbf{w})}
\right\| \leq \phi,
\left\|
\frac{\partial \mathcal{S}_j(\mathbf{w})}{\partial \mathbf{w}} \big{|}_{\mathbf{w}^{(t)}}
\right\| \leq \zeta,
\left\|
\frac{\partial^2 \mathcal{V}_j(\Theta)}{\partial ^2 \Theta} \Big{|}_{\Theta^{(t)}}
\right\| \leq \mathcal{B}$存在,上述等式(不等式)成立。
公式~(\ref{eq:eq_secondOrder})右侧的第二项可以写成如下形式:
\begin{equation}
    \begin{aligned}
        \label{eq:app_second_secondTerm}
          \left\| (G_{ij}) \frac{\partial^2 \mathcal{V}_j(\Theta)}{\partial ^2 \Theta} \Big{|}_{\Theta^{(t)}}  \right\| 
         % = & \left\| 
         % \frac{\partial H^{\rm meta}(\mathbf{\hat w})}{\partial {\mathbf{\hat  w}}} \big{|}_{\mathbf{\hat  w}^{(t)}}^T 
         % \frac{\partial \ell_{\mathbf{x}_j}(\mathcal{S}_j(\mathbf{w}))}{\partial \mathcal{S}_j(\mathbf{w})} \, \frac{\partial \mathcal{S}_j(\mathbf{w})}{\partial \mathbf{w}} \big{|}_{\mathbf{w}^{(t)}} \frac{\partial^2 \mathcal{V}_j(\Theta)}{\partial ^2 \Theta} \Big{|}_{\Theta^{(t)}} 
         % \right\|   \\
          = & \left\| 
         \frac{\partial H^{\rm meta}(\mathbf{\hat w})}{\partial {\mathbf{\hat  w}}} \big{|}_{\mathbf{\hat  w}^{(t)}}^T 
         \frac{\partial \ell_{\mathbf{x}_j}(\mathcal{S}_j(\mathbf{w}))}{\partial \mathcal{S}_j(\mathbf{w})} \, \frac{\partial \mathcal{S}_j(\mathbf{w})}{\partial \mathbf{w}} \big{|}_{\mathbf{w}^{(t)}} \frac{\partial^2 \mathcal{V}_j(\Theta)}{\partial ^2 \Theta} \Big{|}_{\Theta^{(t)}} 
         \right\| \\
         \leq & \, \rho \phi \zeta \mathcal{B}.
    \end{aligned}
\end{equation}
结合公式~(\ref{eq:app_second_firstTerm})和公式~(\ref{eq:app_second_secondTerm}),则有:
\begin{equation}
\left\|
    \nabla_{\Theta^2}^2 H_i^{\rm meta}(\mathbf{\hat w}^{(t)}(\Theta)) \Big{|}_{\Theta^{(t)}}
    \right\|  \leq 
    \phi \zeta (\alpha L \delta^2 \phi \zeta + \rho \mathcal{B}).
\end{equation}
令${\hat L} = \phi \zeta (\alpha L \delta^2 \phi \zeta + \rho \mathcal{B})$,根据拉格朗日中值定理,则有:
\begin{equation}
    \left\| 
    \nabla {L_{\rm meta}(\mathbf{\hat w}^{(t)}(\Theta_1))} - 
    \nabla {L_{\rm meta}(\mathbf{\hat w}^{(t)}(\Theta_2))}
    \right\|
    \leq {\hat L} 
     \left\| 
        \Theta_1 - \Theta_2
        \right\|, \, \text{for all} \, \Theta_1, \Theta_2,
\end{equation}
其中$\nabla {L_{\rm meta}(\mathbf{\hat w}^{(t)}(\Theta_1))} = \nabla_\Theta {L_{\rm meta}(\mathbf{\hat w}^{(t)}(\Theta))}\Big{|}_{\Theta_1}$。
\end{proof}



\section{收敛性证明}
\esection{Proof of Convergence}
\label{chap6_2}


\begin{proof}
元网络的参数在第$t$步中的更新可以写成:
\begin{equation}
     \Theta^{(t+1)} = \Theta^{(t)} - \beta \frac{1}{n} \sum\limits_{i=1}^n
     \nabla_\Theta H_i^{\rm meta}(\mathbf{\hat w}^{(t)}(\Theta)) \Big{|}_{\Theta^{(t)}}. 
\end{equation}
在元数据集中均匀采样出一个小批量的元数据 $\mathrm{B}_t$,上述式子可以写成:
\begin{equation}
     \Theta^{(t+1)} = \Theta^{(t)} - \beta_t \Big{[}
     \sum\limits_{i=1}^n
     \nabla_\Theta H_i^{\rm meta}(\mathbf{\hat w}^{(t)}(\Theta)) + \varepsilon^{(t)} \Big{]}, 
\end{equation}
其中,$\varepsilon^{(t)} =  \nabla_\Theta H^{\rm meta}(\mathbf{\hat w}^{(t)}(\Theta)) \Big{|}_{{\rm B}_t} - \nabla_\Theta H^{\rm meta}(\mathbf{\hat w}^{(t)}(\Theta))$。值得注意的是$\varepsilon^{(t)}$的期望满足$\mathbbm{E}[\varepsilon^{(t)}]=0$,且其方差是有限的。
现在考虑
\begin{equation}
   \begin{aligned}
   \label{eq:eq26}
       & H^{\rm meta}(\mathbf{\hat w}^{(t+1)}(\Theta^{(t+1)})) - 
    H^{\rm meta}(\mathbf{\hat w}^{(t)}(\Theta^{(t)})) \\
    = &  \, \underbrace{H^{\rm meta}(\mathbf{\hat w}^{(t+1)}(\Theta^{(t+1)})) - 
    H^{\rm meta}(\mathbf{\hat w}^{(t)}(\Theta^{(t +1 )}))}_{\rm term 1} +
    \underbrace{H^{\rm meta}(\mathbf{\hat w}^{(t)}(\Theta^{(t+1)})) - 
    H^{\rm meta}(\mathbf{\hat w}^{(t)}(\Theta^{(t)}))}_{\rm term 2}.
   \end{aligned}
\end{equation}

对于公式~(\ref{eq:eq26})中的${\rm term 1}$,由于定理\ref{lam:lemma1},则有:
\begin{equation}
    \begin{aligned}
        & H^{\rm meta}(\mathbf{\hat w}^{(t+1)}(\Theta^{(t+1)})) - 
    H^{\rm meta}(\mathbf{\hat w}^{(t)}(\Theta^{(t+1)})) \\
    \leq & \, 
    \left \langle  
    \nabla H^{\rm meta}(\mathbf{\hat w}^{(t)}(\Theta^{(t+1)})), \mathbf{\hat w}^{(t+1)}(\Theta^{(t+1)}) - \mathbf{\hat w}^{(t)}(\Theta^{(t+1)}) \right \rangle + \frac{L}{2}
    \left \|  
     \mathbf{\hat w}^{(t+1)}(\Theta^{(t+1)}) - \mathbf{\hat w}^{(t)}(\Theta^{(t+1)})
    \right \|_2^2.
    \end{aligned}
\end{equation}
根据公式\ref{eq:step1},\ref{eq:step3},\ref{eq:app_2new},$\mathbf{\hat w}^{(t+1)}(\Theta^{(t+1)}) - \mathbf{\hat w}^{(t)}(\Theta^{(t+1)}) = -\alpha_t \frac{1}{n}\sum\nolimits_{i=1}^n \mathcal{S}(\mathbf{w}^{(t+1)}; \Theta^{(t+1)}) \nabla_{\mathbf{w}} H(\mathbf{w}) \Big{|}_\mathbf{w}^{(t+1)}$成立。则有:
\begin{equation}
\label{eq:term1}
    \left\|
     H^{\rm meta}(\mathbf{\hat w}^{(t+1)}(\Theta^{(t+1)})) - 
    H^{\rm meta}(\mathbf{\hat w}^{(t)}(\Theta^{(t +1 )}))    \right\|
    \leq \alpha_t \rho^2 + \frac{1}{2} L \alpha_t \rho^2
    = \alpha \rho^2 (1 + \frac{\alpha_t L}{2})
\end{equation}
由于$\left\| \frac{\partial H_j(\mathbf{w})}{\partial \mathbf{w}} \big{|}_{\mathbf{\hat w}^{(t)}}\right\| \leq \rho, 
\left\| \frac{\partial H_i^{\rm meta}(\mathbf{w})}{\partial \mathbf{\hat w}} \big{|}_{\mathbf{\hat w}^{(t)}}^T \right\| \leq \rho$,上述式子成立。


对于公式~(\ref{eq:eq26})中的${\rm term 2}$,由于定理\ref{lam:lemma1}中证明了$\nabla H_{\rm meta}(\mathbf{\hat w}^{(t)}(\Theta))$是Lipschitz连续的。则有:
\begin{equation}
    \begin{aligned}
    \label{eq:term2}
    & H^{\rm meta}(\mathbf{\hat w}^{(t)}(\Theta^{(t+1)})) - 
    H^{\rm meta}(\mathbf{\hat w}^{(t)}(\Theta^{(t)})) \\
    \leq & \,  \left \langle  
    H^{\rm meta}(\mathbf{\hat w}^{(t)}(\Theta^{(t+1)})) - 
    H^{\rm meta}(\mathbf{\hat w}^{(t)}(\Theta^{(t)})), \Theta^{(t+1)} - \Theta^{(t)}
     \right \rangle  + \frac{L}{2} 
    \left\| \Theta^{(t+1)} - \Theta^{(t)} \right\|_2^2  \\
    = & \left \langle  
    H^{\rm meta}(\mathbf{\hat w}^{(t)}(\Theta^{(t)})), -\beta_t 
    \Big{[}  H^{\rm meta}(\mathbf{\hat w}^{(t)}(\Theta^{(t)})) + \varepsilon^{(t)}
    \Big{]}
     \right \rangle  + \frac{L \beta_t^2}{2} \left\| \nabla H^{\rm meta}(\mathbf{\hat w}^{(t)}(\Theta^{(t)})) + \varepsilon^{(t)} \right\|_2^2 \\
     = & -(\beta_t - \frac{L \beta_t^2}{2}) \left\| \nabla H^{\rm meta}(\mathbf{\hat w}^{(t)}(\Theta^{(t)})) \right\|_2^2 +
     \frac{L \beta_t^2}{2} \left\| \varepsilon^{(t)} \right\|_2^2 - (\beta_t - L\beta_t^2) \left\langle H^{\rm meta}(\mathbf{\hat w}^{(t)}(\Theta^{(t)})), \varepsilon^{(t)}  \right\rangle.
    \end{aligned}
\end{equation}
累加上述两个公式~(\ref{eq:term1}), ~(\ref{eq:term2}),公式~(\ref{eq:eq26})可以被写成
\begin{equation}
    \begin{aligned}
      &  H^{\rm meta}(\mathbf{\hat w}^{(t+1)}(\Theta^{(t+1)})) - 
    H^{\rm meta}(\mathbf{\hat w}^{(t)}(\Theta^{(t)})) \\
    \leq & \, \alpha \rho^2 (1 + \frac{\alpha_t L}{2}) - (\beta_t - \frac{L \beta_t^2}{2}) \left\| \nabla H^{\rm meta}(\mathbf{\hat w}^{(t)}(\Theta^{(t)})) \right\|_2^2 +
     \frac{L \beta_t^2}{2} \left\| \varepsilon^{(t)} \right\|_2^2 - \\ & \quad\quad\quad\quad\quad\quad\quad (\beta_t - L\beta_t^2) \left\langle H^{\rm meta}(\mathbf{\hat w}^{(t)}(\Theta^{(t)})), \varepsilon^{(t)}  \right\rangle. 
    \end{aligned}
\end{equation}
对上述式子进行整合,则有
\begin{equation}
    \begin{aligned}
        & (\beta_t - \frac{L \beta_t^2}{2}) \left\| \nabla H^{\rm meta}(\mathbf{\hat w}^{(t)}(\Theta^{(t)})) \right\|_2^2  \\
        \leq & \, \alpha \rho^2 (1 + \frac{\alpha_t L}{2}) - (\beta_t - \frac{L \beta_t^2}{2}) \left\| \nabla H^{\rm meta} \big{(} \mathbf{\hat w}^{(t)}(\Theta^{(t)}) \big{)} \right\|_2^2 +  \frac{L \beta_t^2}{2} \left\| \varepsilon^{(t)} \right\|_2^2 - \\ &  \quad\quad\quad\quad\quad(\beta_t - L\beta_t^2) \left\langle H^{\rm meta}(\mathbf{\hat w}^{(t)}(\Theta^{(t)})), \varepsilon^{(t)}  \right\rangle. 
    \end{aligned}
\end{equation}
基于上述不等式,对两侧进行整合,则有
\begin{equation}
    \begin{aligned}
    \label{eq:eq32}
        &  \sum\limits_{t=1}^T (\beta_t - \frac{L \beta_t^2}{2}) \left\| \nabla H^{\rm meta}(\mathbf{\hat w}^{(t)}(\Theta^{(t)})) \right\|_2^2
        \\
        \leq & \, H^{\rm meta}(\mathbf{\hat w}^{(1)}(\Theta^{(1)})) 
        - 
        H^{\rm meta}(\mathbf{\hat w}^{(t)}(\Theta^{(t)}))  + \\
        & \quad \quad \quad \sum\limits_{t=1}^T \alpha \rho^2 (1 + \frac{\alpha_t L}{2}) -
        \sum\limits_{t=1}^T (\beta_t - L\beta_t^2) \left\langle H^{\rm meta}(\mathbf{\hat w}^{(t)}(\Theta^{(t)})), \varepsilon^{(t)}  \right\rangle 
        \, + \,
        \frac{L}{2} \sum\limits_{t=1}^T \left\| \varepsilon^{(t)} \right\|_2^2 \\
        \leq & \, H^{\rm meta}(\mathbf{\hat w}^{(1)}(\Theta^{(1)}))  + \sum\limits_{t=1}^T \alpha \rho^2 (1 + \frac{\alpha_t L}{2}) -
        \sum\limits_{t=1}^T (\beta_t - L\beta_t^2) \left\langle H^{\rm meta}(\mathbf{\hat w}^{(t)}(\Theta^{(t)})), \varepsilon^{(t)}  \right\rangle 
        +
        \frac{L}{2} \sum\limits_{t=1}^T \left\| \varepsilon^{(t)} \right\|_2^2.
    \end{aligned}
\end{equation}
将公式~(\ref{eq:eq32})两侧对$\varepsilon^{(N)}$求期望,则有
\begin{equation}
    \begin{aligned}
 \sum\limits_{t=1}^T (\beta_t - \frac{L \beta_t^2}{2}) \mathop{\mathbbm{E}}\limits_{\varepsilon^{(N)}} \left\| \nabla H^{\rm meta}(\mathbf{\hat w}^{(t)}(\Theta^{(t)})) \right\|_2^2 
        \leq   H^{\rm meta}(\mathbf{\hat w}^{(1)}(\Theta^{(1)})) + \sum\limits_{t=1}^T \alpha \rho^2 (1 + \frac{\alpha_t L}{2}) + \frac{L\sigma^2}{2} \sum\limits_{t=1}^T \beta_t^2,
    \end{aligned}
\end{equation}
因为$\mathop{\mathbbm{E}}\limits_{\varepsilon^{(N)}} \left\langle H^{\rm meta}(\mathbf{\hat w}^{(t)}(\Theta^{(t)})), \varepsilon^{(t)}  \right\rangle=0$ 和 $\mathbbm{\left\| \varepsilon^{(t)} \right\|_2^2} \leq \sigma^2$存在,其中$\sigma^2$ 表示$\varepsilon^{(t)}$的方差。

最终,通过推断可得
\begin{equation}
    \begin{aligned}
        \mathop{\min}\limits_{t} & \, \mathbbm{E} \Big{[} \left\| \nabla H^{\rm meta}(\mathbf{\hat w}^{(t)}(\Theta^{(t)})) \right\|_2^2  \Big{]}  \leq
        \frac{\sum\nolimits_{t=1}^T (\beta_t - \frac{L \beta_t^2}{2}) \mathop{\mathbbm{E}}\limits_{\varepsilon^{(N)}} \left\| \nabla H^{\rm meta}(\mathbf{\hat w}^{(t)}(\Theta^{(t)})) \right\|_2^2 }{\sum\nolimits_{t=1}^T (\beta_t - \frac{L \beta_t^2}{2})}  \\
        \leq & \, \frac{1}{\sum\nolimits_{t=1}^T (2\beta_t - L\beta_t^2)} \Big{[}
        2H^{\rm meta}(\mathbf{\hat w}^{(1)}(\Theta^{(1)})) + \sum\limits_{t=1}^T \alpha \rho^2 (2 + \alpha_t L) + L\sigma^2 \sum\limits_{t=1}^T \beta_t^2
        \Big{]} \\
        \leq & \, \frac{1}{\sum\nolimits_{t=1}^T \beta_t}  \Big{[}
        2H^{\rm meta}(\mathbf{\hat w}^{(1)}(\Theta^{(1)})) + \sum\limits_{t=1}^T \alpha \rho^2 (2 + \alpha_t L) + L\sigma^2 \sum\limits_{t=1}^T \beta_t^2
        \Big{]} \\
        \leq & \, \frac{1}{T \beta_t} \Big{[}
        2H^{\rm meta}(\mathbf{\hat w}^{(1)}(\Theta^{(1)})) +  
        \alpha_1 \rho^2 T (2 +  L) + 
        L\sigma^2 \sum\limits_{t=1}^T \beta_t^2
        \Big{]} \\
        = & \, \frac{2H^{\rm meta}(\mathbf{\hat w}^{(1)}(\Theta^{(1)}))}{T} \, \frac{1}{\beta_t} + \frac{2  \alpha_1 \rho^2  (2 +  L)}{\beta_t}
        + \frac{L\sigma^2}{T} \sum\limits_{t=1}^T \beta_t \\
        \leq & \, \frac{2H^{\rm meta}(\mathbf{\hat w}^{(1)}(\Theta^{(1)}))}{T} \, \frac{1}{\beta_t} + \frac{2  \alpha_1 \rho^2  (2 +  L)}{\beta_t} + L\sigma^2  \beta_t \\
        = & \, \frac{H^{\rm meta}(\mathbf{\hat w}^{(1)}(\Theta^{(1)}))}{T} \max\{L, \frac{\sigma\sqrt{T}}{\rm c}\} + \min\{1, \frac{k}{T}\}\max\{L, \frac{\sigma\sqrt{T}}{\rm c}\}\rho^2(2+L) + L\sigma^2 \min\{\frac{1}{L}, \frac{\rm c}{\sigma\sqrt{T}}\} \\
        \leq & \, \frac{\sigma H^{\rm meta}(\mathbf{\hat w}^{(1)}(\Theta^{(1)}))}{{\rm c} \sqrt{T}} + \frac{k\sigma\rho^2(2+L)}{{\rm c} \sqrt{T}} + \frac{L\sigma{\rm c}}{\sqrt{T}} = \mathcal{O}(\frac{1}{\sqrt{T}}).
    \end{aligned}
\end{equation}
因此,当满足一定温和条件时,本文的学习算法能在$T$步内实现$\min_{0 \leq t \leq T} \mathbbm{E} \Big{[} \left\| \nabla H^{\rm meta}(\Theta^{(t)}) \right\|_2^2  \Big{]} \leq \mathcal{O}(\frac{1}{\sqrt{T}})$。

\end{proof}











\SDUbackmatter

%%%%%%%%%%%%%%%%%%%%%%%%%%%%%%
%% 参考文献
%%%%%%%%%%%%%%%%%%%%%%%%%%%%%%
\ereference{Reference}
\bibliographystyle{gbt7714-numerical}
\bibliography{bib}

%%%%%%%%%%%%%%%%%%%%%%%%%%%%%%%
%%% 致谢页
%%%%%%%%%%%%%%%%%%%%%%%%%%%%%%%
%\cleardoublepage
\begin{thanks}
\ethanks{Acknowledgements}
时光荏苒,岁月如梭,不知不觉我的三年研究生生涯即将结束。从进入山东大学攻读硕士学位开始,时至今日,我走过了一段意义非凡的旅程。这一路上,承蒙多位师友的帮助,收获颇丰。值此论文完成之际,谨以诚挚之心向他们一一表示感谢。
\end{thanks}


%%%%%%%%%%%%%%%%%%%%%%%%%%%%%%%
%%% 发表论文目录
%%%%%%%%%%%%%%%%%%%%%%%%%%%%%%%
\begin{publications}
\epublications{Catalogue of Academic Papers Published During Degree Study}
\begin{enumerate}

\item 
XXX. YYY, 2025, 15(1): 379. (\textbf{已发表},第一作者,SCI期刊)
 
\item  
XXXX. YYYY. (\textbf{已录用},第一作者,CCF-C)  
\end{enumerate}
\end{publications}

%\cleardoublepage
% \input{./Chapters/projects}
%%%%%%%%%%%%%%%%%%%%%%%%%%%%%%%
%%% 学位论文评阅及答辩情况表,两种方式
%%%%%%%%%%%%%%%%%%%%%%%%%%%%%%%
\includepdfmerge{./Chapters/论文评阅及答辩情况表扫描件.pdf,-} %需要等ending paper下发然后添加 by susie
% \includepdfmerge{./Chapters/page_last.pdf,-} 



\end{document}